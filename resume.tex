\documentclass[]{awesome-cv}
\usepackage{tikz}
\geometry{left=1.5cm, top=1cm, right=1.0cm, bottom=1cm, footskip=.3cm}
\newcommand{\ExternalLink}{%
    \tikz[x=1ex, y=1ex, baseline=-0.05ex]{% 
        \begin{scope}[x=1ex, y=1ex]
            \clip (-0.1,-0.1) 
                --++ (-0, 1.2) 
                --++ (0.6, 0) 
                --++ (0, -0.6) 
                --++ (0.6, 0) 
                --++ (0, -1);
            \path[draw, 
                line width = 0.5, 
                rounded corners=0.5] 
                (0,0) rectangle (1,1);
        \end{scope}
        \path[draw, line width = 0.5] (0.5, 0.5) 
            -- (1, 1);
        \path[draw, line width = 0.5] (0.6, 1) 
            -- (1, 1) -- (1, 0.6);
        }
    }
\usepackage{textcomp}
\usepackage{hyperref}
\fontdir[fonts/]
\newcommand*{\sectiondir}{resume/}
\colorlet{awesome}{awesome-red}

 \begin{document}
\begin{flushleft}

	\headerlastnamestyle{Маслова } \headerfirstnamestyle{ Анна Олеговна} \\
	\vspace{1mm}

\end{flushleft}


\vspace{1mm}
\cvsection{Персональная информация}
\begin{cventries}
	\cventry
	{}
	{\def\arraystretch{1.5}{\begin{tabular}{ l  l }
		День рождения:  & {\qquad\skill{18.12.1999}} \\
		Телефон:  & {\qquad\skill{7 (916) 743-70-67}} \\
		Почта:      &{\qquad\skill{\href{mailto:maslovaa@yahoo.com}{maslovaa@yahoo.com}}    \ExternalLink} \\
		Телеграм:  & {\qquad\skill{@anbananova}} \\
		\end{tabular}}}
	{}
	{}
	{}
\end{cventries}

\vspace{-9mm}
\cvsection{Образование}
\begin{cventries}
	\cventry
	{Бакалавриат, ПМиИТ, Прикладная математика и информатика \newline GPA (средний балл) 4.8/5, 3-ий курс}
	{Финансовый университет при Правительстве Российской Федерации}
	{Москва, 2017 – 2021}
	{}
	{}
	\end{cventries}  \vspace{-6mm} \begin{cventries}
	\cventry
	{Социально-экономический профиль \newline Окончила с золотой медалью "За особые успехи в обучении"}
	{ГБОУ "Лицей № 1575"}
	{Москва, 2014 – 2017}
	{}
	{}
\end{cventries}
\vspace{-5mm}


\cvsection{Опыт работы}
\begin{cventries}
	\cventry
	{Волонтер \newline \quad \bullet  Занималась ведением архива: организовывала хранение документов, составляла личные дела абитуриентов}
	{Приемная комиссия Финансового университета при Правительстве РФ}
	{Москва, 2019}
	{}
	{}
	\end{cventries}
	
\vspace{-5mm}	
\cvsection{Навыки}
\vspace{-2mm}
\begin{cventries}
	\cventry
	{}
	{\def\arraystretch{1.15}{\begin{tabular}{ l l }
		Английский: & {\qquad\skill{C1 / Advanced}} \\
		Python:  & {\qquad\skill{Программирую и анализирую данные с помощью  pandas и numpy (библиотеки алгоритмов}} \\
		   & {\qquad\skill{для анализа, моделирования и обработки данных, многомерных массивов)}} \\
		R:  & {\qquad\skill{Программирую на среднем уровне}} \\
		SQL: & {\qquad\skill{Знаю систему и язык запросов (Microsoft SQL Server)}} \\
		%\skill{\LaTeX\ } & {\qquad\skill{На уровне, необходимом для создания такого резюме}} \\
		%1С:Бухгалтерия, 1С:Предприятие:& {\qquad\skill{Знание основных функций систем}} \\
		\end{tabular}}}
	{}
	{}
	{}
\end{cventries}

\vspace{-9mm}
\cvsection{Проекты}
\begin{cventries}
	\cventry
	{1) \textbf{"Анализ  недобросовестных практик на рынке ценных бумаг за 2013-2017 гг"}
	\newline \qquad \bullet  Разработала алгоритм выявления и предотвращения негативных явлений и последствий, возникающих в результате недобросовестных практик на рынке ценных бумаг, запрограммировала и провела его проверку на основе статистических данных правоприменительной деятельности Центрального Банка России
	\newline 
	\textbf{С работой участвовала в следующих мероприятиях:}
	\newline \quad IX Грушинская социологическая конференция  
	\newline \quad Интерактивный стендовый конкурс-выставка "Турнир научных идей"  
	\newline \quad Круглый стол «Цифровой ландшафт нашего экономического будущего» в рамках Х 
	\newline \quad Международного научного студенческого конгресса
	\newline \textbf{Победила в стендовом конкурсе бизнес-проектов} } 
	{Научно-исследовательские работы:}
	{Москва, 2019}
	{}
	{}
		\end{cventries}  \vspace{-6mm} \begin{cventries}
	\cventry
	{2) \textbf{"Методы анализа случайных рисков с помощью центральных и 
промежуточных порядковых статистик"}
	\newline \qquad \bullet Проверила и на статистике стихийных бедствий в странах Западной Европы подтвердила гипотезу о том, что с помощью сравнительного анализа медианы и среднего значения выборки, можно сделать вывод о том, являются ли данные случайными рисками
	}
	{}%Написание курсовой научно-исследовательской работы}
	{}
	{}
	{}
	\vspace{-5mm}
\end{cventries}


\cvsection{Сертификаты}
\begin{cventries}
	\cventry
		{}
	{Прошла курс Bloomberg Market Concepts:
	\newline \quad \bullet BMC's Core Concepts course
	\newline \quad \bullet BMC Getting Started on the Terminal
	\newline \quad \bullet BMC Portfolio Management
	\newline \quad и получила соответствующий сертификат (Certificate ID: 157016797541)}
	{}
	{}
	{}
	\end{cventries} \vspace{-6mm} \begin{cventries}
	\cventry
{}
	{Прошла курс повышения финансовой грамотности в рамках VI Всероссийской недели сбережений и получила соответствующий сертификат}
	{}
	{}
	{}
	\end{cventries} 
	\vspace{-7mm}
	
\cvsection{Прочее}
\begin{cventries}
	\cventry
	{\qquad \bullet На курсе я улучшила свою способность справляться с вызовами окружающей среды, развила навыки концентрации, стала более эмоционально уравновешенной}
	{Прошла обучение по курсу осознанности Mindfulness }
	{Лаборатория RealMindfulness}
	{}
	{}
	\end{cventries}  \vspace{-7mm} \begin{cventries}
	\cventry	
	{\qquad \bullet Обучилась методу слепой печати}
	{Прошла курс машинописи}
	{}
	{}
	{}
		\end{cventries}  \vspace{-7mm}\begin{cventries}
	\cventry 
	{}
	{Именной стипендиат Правительства Москвы}
	{}{}{}
		\end{cventries}  \vspace{-11mm} \begin{cventries}
	\cventry	
	{}
	{Золотой значок ГТО (V ступень)}
	{}
	{}
	{}
		\end{cventries}  \vspace{-11mm} \begin{cventries}
	\cventry
	{\qquad Неоднократный призер и победитель международных и общероссиских турниров}
	{Кандидат в мастера спорта по спортивным бальным танцам}
	{}
	{}
	{}
	\end{cventries} \vspace{-10mm}
\end{document}
