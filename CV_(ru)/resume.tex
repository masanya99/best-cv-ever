\documentclass[]{awesome-cv}
\usepackage{tikz}
\geometry{left=1cm, top=1cm, right=1.0cm, bottom=0.2cm, footskip=.3cm}
\newcommand{\ExternalLink}{%
    \tikz[x=1ex, y=1ex, baseline=-0.05ex]{% 
        \begin{scope}[x=1ex, y=1ex]
            \clip (-0.1,-0.1) 
                --++ (-0, 1.2) 
                --++ (0.6, 0) 
                --++ (0, -0.6) 
                --++ (0.6, 0) 
                --++ (0, -1);
            \path[draw, 
                line width = 0.5, 
                rounded corners=0.5] 
                (0,0) rectangle (1,1);
        \end{scope}
        \path[draw, line width = 0.5] (0.5, 0.5) 
            -- (1, 1);
        \path[draw, line width = 0.5] (0.6, 1) 
            -- (1, 1) -- (1, 0.6);
        }
    }
\usepackage{textcomp}
\usepackage{hyperref}
\fontdir[fonts/]
\newcommand*{\sectiondir}{resume/}
\colorlet{awesome}{awesome-red}
\usepackage{xcolor}


 \begin{document}
\begin{flushleft}

	\headerlastnamestyle{Маслова } \headerfirstnamestyle{ Анна Олеговна} \\
	\vspace{1mm}

\end{flushleft}

\vspace{1mm}
\cvsection{Персональная информация}
\begin{cventries}
	\cventry
	{}
	{\def\arraystretch{1.5}{\begin{tabular}{ l  l }
		День рождения:  & {\qquad\skill{18.12.1999}} \\
		Телефон:  & {\qquad\skill{7 (916) 743-70-67}} \\
		Почта:      &{\qquad\skill{\href{mailto:maslovaa@yahoo.com}{maslovaa@yahoo.com}}    \ExternalLink} \\
		Телеграм:  & {\qquad\skill{\href{https://t-do.ru/anbananova}{@anbananova}}     \ExternalLink} \\
		GitHub:  & {\qquad\skill{\href{https://github.com/masanya99}{@masanya99}}     \ExternalLink} \\
		\end{tabular}}}
	{}
	{}
	{}
\end{cventries}

\vspace{-11mm}
\cvsection{Образование}
\begin{cventries}
	\cventry
	{Бакалавриат, ИТиАБД, Прикладная математика и информатика \newline GPA (средний балл) 4.9/5, 4-ый курс}
	{Финансовый университет при Правительстве Российской Федерации}
	{Москва, 2017 – 2021}
	{}
	{}
	\end{cventries}  \vspace{-6mm} \begin{cventries}
	\cventry
	{Социально-экономический профиль \newline Окончила с золотой медалью "За особые успехи в обучении"}
	{ГБОУ "Лицей № 1575"}
	{Москва, 2014 – 2017}
	{}
	{}
\end{cventries}
\vspace{-7mm}


\cvsection{Опыт работы}
\begin{cventries}
	\cventry
	{Стажер IT-отдела \newline \quad \bullet  Занималась 1С-разработкой: писала запросы для документов}
	{АО «Фаберлик», Департамент информационных технологий}
	{Москва, 2020}
	{}
	{}
	\end{cventries}  \vspace{-6mm}
	\begin{cventries}
	\cventry
	{Волонтер \newline \quad \bullet  Занималась ведением архива: организовывала хранение документов, составляла личные дела абитуриентов}
	{Приемная комиссия Финансового университета при Правительстве РФ}
	{Москва, 2019}
	{}
	{}
	\end{cventries}
	\vspace{-7mm}	
	
\cvsection{Навыки}
\vspace{-2mm}
\begin{cventries}
	\cventry
	{}
	{\def\arraystretch{1.15}{\begin{tabular}{ l l }
		Английский: & {\qquad\skill{C1 / Advanced}} \\
		Python:  & {\qquad\skill{Программирую и анализирую данные с помощью pandas,  numpy,  dask,  nltk,  xlwings,  scipy,  seaborn, }} \\
		   & {\qquad\skill{sklearn, PyTorch (библиотеки алгоритмов для анализа, моделирования и обработки данных, }} \\
		   & {\qquad\skill{многомерных массивов,  а также библиотеки машинного обучения и прогноза данных)}} \\
		R:  & {\qquad\skill{Программирую на среднем уровне}} \\
		SQL: & {\qquad\skill{Знаю систему и язык запросов (Microsoft SQL Server)}} \\
		MS Excel: & {\qquad\skill{Могу строить и анализировать финансовые модели}} \\
		1С:Предприятие:& {\qquad\skill{Знаю основные функции системы}} \\
		SAP S/4HANA:& {\qquad\skill{Знаю основные функции и особенности системы}} \\
		\end{tabular}}}
	{}
	{}
	{}
\end{cventries}

\vspace{-11mm}
\cvsection{Проекты}
\begin{cventries}
	\cventry
	{\textbf{"Анализ  недобросовестных практик на рынке ценных бумаг за 2013-2017 гг"}
	\newline \qquad \bullet  Разработала алгоритм выявления и предотвращения негативных явлений и последствий, возникающих в результате недобросовестных практик на рынке ценных бумаг, запрограммировала и провела его проверку на основе статистических данных правоприменительной деятельности Центрального Банка России
	\newline 
	\textbf{С работой участвовала в следующих мероприятиях:}
	\newline \quad IX Грушинская социологическая конференция  
	\newline \quad Интерактивный стендовый конкурс-выставка "Турнир научных идей"  
	\newline \quad Круглый стол «Цифровой ландшафт нашего экономического будущего» в рамках Х 
	\newline \quad Международного научного студенческого конгресса
	\newline \textbf{Победила в стендовом конкурсе бизнес-проектов} } 
	{Научно-исследовательская работа:}
	{2019}
	{}
	{}
		\end{cventries}  \vspace{-6mm} \begin{cventries}
	\cventry
	{1) \textbf{"Технологии создания и применения чат-ботов"}
	\newline \qquad \bullet Исследовала технологии создания чат-ботов и их применения в реальной жизни.  Рассмотрела историю появления феномена чат-ботов,  исследовала способы их создания.  В работе привела примеры актуальных чат-ботов из различных сфер жизнедеятельности. Также показала ботов, которые применяются в отделах кадров компаний.  В конце работы разобрала библиотеку для создания чат-ботов в мессенджере Telegram и разработала свой чат-бот. 
	}
	{Курсовые работы:}
	{2019-2020}
	{}
	{}
\end{cventries}
\vspace{-5mm} \begin{cventries}
	\cventry
	{2) \textbf{"Методы анализа случайных рисков с помощью центральных и 
промежуточных порядковых статистик"}
	\newline \qquad \bullet Проверила и на статистике стихийных бедствий в странах Западной Европы подтвердила гипотезу о том, что с помощью сравнительного анализа медианы и среднего значения выборки, можно сделать вывод о том, являются ли данные случайными рисками.
	}
	{}
	{}
	{}
	{}
	\vspace{-7mm}
\end{cventries}


\cvsection{Курсы и Сертификаты}
\begin{cventries}
	\cventry
		{}
	{Прошла программу обучения SAP S/4HANA Academy}
	{SAP СНГ}
	{декабрь 2020}
	{}
	\end{cventries} \vspace{-5mm}\begin{cventries}
	\cventry
		{}
	{Участвовала в марафоне «Data Science и ИИ: пишем сценарий сериала»}
	{НЕТОЛОГИЯ}
	{ноябрь 2020}
	{}
	\end{cventries} \vspace{-5mm}\begin{cventries}
	\cventry
		{}
	{Прошла интенсив «Профессия Data Scientist: учимся обработке и анализу данных за 3 дня»}
	{Skillbox}
	{октябрь 2020}
	{}
	\end{cventries} \vspace{-5mm}\begin{cventries}
	\cventry
		{}
	{Прошла интенсив «Мессенджер на Python за 3 дня»}
	{Skillbox}
	{август 2020}
	{}
	\end{cventries} \vspace{-5mm}
	%\begin{cventries}
	%\cventry
	%	{}
	%{Прошла интенсив «Станьте хакером на Python за 3 дня»}
	%{Skillbox}
	%{август 2020}
	%{}
	%\end{cventries} \vspace{-5mm}
	\begin{cventries}
	\cventry
		{}
	{Прошла интенсив «Напишите первую модель машинного обучения за 3 дня»}
	{Skillbox}
	{август 2020}
	{}
	\end{cventries} \vspace{-5mm}\begin{cventries}
	\cventry
		{}
	{Прошла интенсив «Чат-бот с искусственным интеллектом на Python»}
	{Skillbox}
	{июль 2020}
	{}
	\end{cventries} \vspace{-5mm}\begin{cventries}
	\cventry
		{}
	{Прошла "Базовый курс для начинающих 1С-программистов" }
	{Infostart}
	{июль 2020}
	{}
	\end{cventries} \vspace{-5mm}\begin{cventries}
	\cventry
	{и получила соответствующий сертификат}
	{Прошла курс "Математика и Python для анализа данных" }
	{Coursera}
	{май 2020}
	{}
	\end{cventries} \vspace{-5mm} \begin{cventries}
	\cventry
{в рамках VI Всероссийской недели сбережений и получила соответствующий сертификат}
	{Прошла курс повышения финансовой грамотности}
	{Сбербанк}
	{декабрь 2019}
	{}
	\end{cventries}\vspace{-5mm}  \begin{cventries}
	
	\cventry
		{\quad \bullet BMC's Core Concepts course
	\newline \quad \bullet BMC Getting Started on the Terminal
	\newline \quad \bullet BMC Portfolio Management
	\newline \quad и получила соответствующий сертификат (Certificate ID: 157016797541)}
	{Прошла курс Bloomberg Market Concepts:}
	{Bloomberg}
	{ноябрь 2019}
	{}
	\end{cventries} 
	\vspace{-7mm}


\cvsection{Хакатоны}
	\begin{cventries}
	\cventry
	{}
	{Участвовала в хакатоне "More Tech: Web Track"}
	{ВТБ}
	{октябрь 2020}
	{}
	\end{cventries}  \vspace{-5mm} 
	%\begin{cventries}
	%\cventry
	%{}
	%{Участвовала в Онлайн-Хакатоне "Прокачай Бизнес"}
	%{VISA, ikra}
	%{август 2020}
	%{}
	%\end{cventries}  \vspace{-5mm}
	 \begin{cventries}
	\cventry	
	{}
	{Участвовала в Математической регате Тинькофф}
	{Тинькофф}
	{июль 2020}
	{}
		\end{cventries}  \vspace{-5mm} \begin{cventries}
	\cventry	
	{}
	{Участвовала в International Data Science Hackathon}
	{McKinsey and Company}
	{май 2020}
	{}
		\end{cventries}  \vspace{-7mm}


\cvsection{Прочее}
\begin{cventries}
	\cventry
	{\qquad \bullet На курсе я улучшила свою способность справляться с вызовами окружающей среды, развила навыки концентрации, стала более эмоционально уравновешенной}
	{Прошла обучение по курсу осознанности Mindfulness }
	{Лаборатория RealMindfulness}
	{}
	{}
	\end{cventries}  \vspace{-7mm} \begin{cventries}
	\cventry	
	{\qquad \bullet Обучилась методу слепой печати}
	{Прошла курс машинописи}
	{}
	{}
	{}
		\end{cventries}  \vspace{-7mm}\begin{cventries}
	\cventry 
	{}
	{Именной стипендиат Правительства Москвы}
	{}{}{}
		\end{cventries}  \vspace{-11mm} \begin{cventries}
	\cventry	
	{}
	{Золотой значок ГТО (V ступень)}
	{}
	{}
	{}
		\end{cventries}  \vspace{-11mm} \begin{cventries}
	\cventry
	{\qquad Неоднократный призер и победитель международных и общероссиских турниров}
	{Кандидат в мастера спорта по спортивным бальным танцам}
	{}
	{}
	{}
	\end{cventries} \vspace{-10mm}
\end{document}
